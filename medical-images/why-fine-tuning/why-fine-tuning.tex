\documentclass[a4paper,12pt]{article}
  \usepackage{inputenc}
  \usepackage{indentfirst}
  \title{Convolutional Neural Networks for Medical Image Analysis: Full Training or Fine Tuning?}
  \author{Nima Tajbakhsh \emph{et al.}}
  \date{2016}

\begin{document}
\maketitle

\section{Contribution}

In this paper, we systematically study knowledge transfer to medical imaging applications, making the following contributions:

\begin{itemize}
	\item We demonstrated how fine-tuning a pre-trained CNN in a layer-wise manner leads to incremental performance im-provement.
	\item We analyzed how the availability of training samples influences the choice between pre-trained CNNs and CNNs trained from scratch.
	\item We compared the performance of pre-trained CNNs, not only against handcrafted approaches but also against CNNs trained from scratch using medical imaging data.
	\item We presented consistent results with conclusive outcomes for 4 distinct medical imaging applications involving clas-sification, detection, and segmentation in 3 different medical imaging modalities.
\end{itemize}

\section{Experiment Result}

When using complete datasets,

\[\textrm{Deep fine-tuning} \ge \textrm{Scratch} > \textrm{Shallow fine-tuning}\]

When using reduced datasets,

\[\textrm{Deep fine-tuning} >> \textrm{Scratch}\]

\section{Conclusion}

Deeply fine-tuned CNNs should always be the preferred option due to following reasons:

\begin{enumerate}
	\item Better performance.
	\item Tolerance to reduced dataset size.
	\item Faster convergence.
\end{enumerate}

\end{document}
